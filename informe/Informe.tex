\documentclass{article}
\usepackage{graphicx}
\usepackage{multicol}
\usepackage{amsmath}
\usepackage{lipsum}

\title{Reconocimiento de señales de tráfico cubanas}
\author {
Claudia Alvarez Martínez \and
Roger Moreno Gutiérrez \and
Kevin Majim Ortega Alvarez \and
Jan Carlos Perez González
}

\date {\today}

\begin{document}
\begin{figure}[t]
    \centering
    \includegraphics{logo matcompng.png}
\end{figure}
\maketitle
\begin{center}
Facultad de Matemática y Computación. Universidad de la Habana
\end{center}
\newpage
\begin{abstract}
La detección y clasificación de señales de tránsito es una tarea para apoyar al conducto e incluso asistir en la navegación de un automóvil autónomo. El objetivo de este trabajo es presentar una metodología para la detección y clasificación de señales de tráfico cubanas mediante el uso de redes convolucionales. La metodología se divide en cinco etapas, 1-) se realiza la recolección de 615 de señales de tráfico en un ambiente no controlado, 2-) se realiza un proceso manual de marcado para la detección de señales de tráfico en las imágenes, 3-) se entrena una red neuronal (YOLOv8) en el conjunto GTSRB para obtener un conocimiento general de las características de las señales, 4-) se evalúa esa red neuronal en el conjunto de datos creado, 5-) se realiza un proceso de transferencia de conocimiento y aumentado de datos para la clasificación. \textbf{AQUI VA MUELA DE LOS RESULTADOS}
\vfill
\textbf{Palabras clave}: Señales de tráfico, redes convolucionales, imágenes segmentadas, YOLOv8
\end{abstract}
\newpage
%\begin{multicols}{2}
\section{Introducción}
Las señales de tráfico son los signos visuales utilizados para ofrecerle información a los conductores y peatones que transitan por un camino. Las señales de tráfico se clasifican de acuerdo a sus colores y formas, de manera que sean llamativas para que los automovilistas les presten atención. Muchos países europeos han trabajado para estandarizar las señales de tráfico, lo que dio lugar a la convención de Viena del 8 de noviembre de 1968. 

Particularmente en nuestro país, las señales de tránsito se distribuyen en 8 grupos\cite{ref7}, entre las que se encuentran algunas de prohibición, que por lo general son blancas y rojas en los bordes, las de peligro son triángulos amarillos o las de obligación que son redondas y de fondo azul.

Es sabido por todos los conductores de vehículos la importancia de obedecer las leyes de tránsito, respetando con especial énfasis las señales de tráfico, dado que las consecuencias pueden ser nefastas cuando las mismas están en un estado deprobable debido a las inclemencias del clima, mala iluminación, o simplemente no se ven a tiempo. De ahí la importancia de contar con sistema automatizado que sea capaz de detectar de forma rápida y eficiente estas señales, además, este sistema puede ser utilizado como asistencia a conductores, mejorando la seguridad en la vía

Debido a que aún se encuentran retos, para diseñar un sistema de detección totalmente exitoso, el presente trabajo tiene como objetivo, desarrollar una metodología para la detección y clasificación de señales cubanas, además de proporcionar una base de datos de las mismas, con imágenes obtenidas en un entorno no controlado; para la evaluación de este tipo de modelos.


\newpage
\section{Revisión bibliográfica}
Dado que las señales de tráfico se clasifican por sus formas y colores, se pueden aplicar técnicas de aprendizaje de máquinas y visión computacional para su reconocimiento \cite{ref6}, aunque las más utilizadas son las redes convolucionales pues estas han demostrado poseer una gran precisión en la detección de señales. A. Hechri y A. Mtibaa \cite{ref8} realizan una comparación entre dos aproximaciones distintas, una utilizando Maquinas de Soporte Vectorial (SVM por sus siglas en inglés) y la otra utilizando redes convolucionales, además proponen un nuevo modelo a dos fases, la primera de detección y reconocimiento, en la cual utilizan el SVM, y la segunda de clasificación, en la cual utiliza redes convolucionales para esta tarea. Se muestran resultados prometedores para este modelo, y un tanto desalentadores para el SVM.

Divya Elankumaran \cite{ref5} propone un algoritmo que combina redes residuales (ResNet-50), YOLOv8 (You Only Look Once) y redes neuronales convolucionales profundas (DCNN) durante el entrenamiento para resolver esta tarea. En él, se utiliza YOLOv8 para la detección de las señales, mientras que de la clasificación se utilizan las DCNN, y para mitigar el efecto que pueda tener el trabajo con estas redes profundas se utiliza ResNet-50, dado que como posee capas residuales, permite que el gradiente salte ciertas capas durante el entrenamiento.

Gege Guo y Zhenyu Zhang \cite{ref9} crean un algoritmo ligero basado en YOLOv5s para la detección de baches en las calles de manera más rápida, y aunque no detecte señales de tránsito, utiliza varias técnicas que son interesantes. Cambian la estructura de YOLOv5s, dado que modifican la columna (Backbone) del mismo utilizando una red MobileNetV3, la cual es una red, eficiente en términos de computación y memoria, lo cual se traduce en una mayor rapidez, aplican el algoritmo de KMeans para determinar automáticamente las mejores anclas ("anchor boxes" son uno de los hiperparámetros de YOLOv5); el suavizado de etiquetas y la reparametrización estructural. en el año 2022, Yanzhao Zhu y Wei Qi Yan \cite{ref1} realizan una comparación entre los modelos de YOLOv5 y SSD (Single Shot MultiBox Detector), donde los resultados evidencian que YOLOv5 posee métricas mejores para el reconocimiento de señales.


En 2023 Weizhen Song and Shahrel Azmin Suani\cite{ref2} proponen un nuevo modelo basado en YOLOv4-tiny y que utiliza una versión simplificada CSPDarknet53 para la extracción de características, Agrupamiento de Pirámides Espaciales Mejorado (spatial-pyramid-pooling) lo cual mejora la capacidad del modelo para procesar objetos de diferentes tamaños; además de utilizar el algoritmo e K-Means para agrupar los datos y encontrar las mejores cajas anclas (anchor boxes).

Para concluir, los métodos basados en aprendizaje profundo, nos ofrecen amplios beneficios, dado su efectividad, precisión y rapidez para detectar y clasificar señales. En particular YOLO, es el que mejores resultados ofrece, además con la llegada de YOLOv8, como son la optimización de bloques residuales y cuellos de botella (bottleneck blocks) más eficientes, optimización de anclas con el algoritmo de K-Means++ mejorando la precisión del modelo, se han mejorado las partes fundamentales de la red (backbone) y la cabeza de la red (head) para una mejor extracción de características y detección de objetos; hace que YOLOv8 sea un buen candidato para utilizar en nuestro modelo.
\section{Nuestro Modelo}
\lipsum[1-4]
\section{Resultados}
\lipsum[1-4]
\section{Conclusiones}
\lipsum[1-4]
\section{Bibliografía}
\bibliographystyle{plain}
\bibliography{referencias} 
%end{multicols}
\end{document}



















